\section{Seitenabmessungen und andere Längen}
Nachfolgend eine Tabelle mit den Abmessungen und Längen welche durch die HSR-Klasse und das HSR-Package festgelegt werden.
Die Längenangaben sind mit Vorsicht zu geniessen, da sie einfach durch eigenen Commands oder Packages überschrieben werden können.
Die Werte in der Tabelle wurden mit \LaTeX generiert, dadurch entstehen relativ grosse Rundungsfehler.

\newdimen\blsk
\blsk=\baselineskip

\begin{table}[!ht]
\centering
\rowcolors{3}{HSRLightGray20}{HSRLightGray40}
\begin{tabular}{*{4}{l}}
	\rowcolor{HSRBasswood40}
	\textbf{Name} 		& \textbf{\LaTeX-Makro}		& \textbf{Länge (cm)} 								& \textbf{Länge (pt)}
	\\
	\rowcolor{HSRLakeGreen40}
	\textbf{Papier und Satz} & & &
	\\
	Papierbreite 		& \verb+\paperwidth+		& \uselengthunit{cm}\printlength{\paperwidth}		& \uselengthunit{pt}\printlength{\paperwidth}
	\\
	Papierhöhe			& \verb+\paperheight+		& \uselengthunit{cm}\printlength{\paperheight}  	& \uselengthunit{pt}\printlength{\paperheight}
	\\
	Textbreite			& \verb+\textwidth+			& \uselengthunit{cm}\printlength{\textwidth}		& \uselengthunit{pt}\printlength{\textwidth}
	\\
	Texthöhe			& \verb+\textheight+		& \uselengthunit{cm}\printlength{\textheight}		& \uselengthunit{pt}\printlength{\textheight}
	\\
	Zeilenabstand$^1$	& \verb+\baselineskip+		& \uselengthunit{cm}\printlength{\blsk}				& \uselengthunit{pt}\printlength{\blsk}
	\\
	\rowcolor{HSRLakeGreen40}
	\textbf{Tabellen} & & &
	\\
	Zellenabstand		& \verb+\columnsep+			& \uselengthunit{cm}\printlength{\columnsep}		& \uselengthunit{pt}\printlength{\columnsep}
	\\
	Zellenabstand$^2$	& \verb+\tabcolsep+			& \uselengthunit{cm}\printlength{\tabcolsep}		& \uselengthunit{pt}\printlength{\tabcolsep}
	\\
	Breite Trennlinie	& \verb+\arrayrulewidth+	& \uselengthunit{cm}\printlength{\arrayrulewidth}	& \uselengthunit{pt}\printlength{\arrayrulewidth}					 
	
\end{tabular}
\caption{Diverse Längenangaben}
\label{tab:laengen}
\end{table}

\noindent
\textbf{Anmerkungen zu Tabelle \ref{tab:laengen}:} \\
$^1$ Der \verb+\baselinestretch+ wird nicht geändert und bleibt daher 0! \\
$^2$ Welches ist nun die richtige Länge?
